% vim:encoding=utf8 ft=tex sts=2 sw=2 et:

\documentclass{classrep2}
\usepackage[utf8]{inputenc}
\usepackage{graphicx}

\studycycle{Informatyka, studia dzienne, mgr jednolite}
\coursesemester{IX}

\coursename{Obliczenia ewolucyjne}
\courseyear{2010/2011}

\courseteacher{mgr inż. Łukasz Chomątek}
\coursegroup{wtorek, 14:30}

\author{
  \studentinfo{Cezar Pokorski}{138077} \and
  \studentinfo{Artur Czajka}{137971} 
}

\title{Zadanie 4: Całkowanie przez GP}
\hgurl{https://aelabnull.googlecode.com/hg/4/}

\begin{document}
\maketitle

\section{Istota zadania}
Dla zadanej całki nieoznaczonej należy ewolucyjnie wygenerować formułę symboliczną reprezentującą funkcję pierwotną metodami programowania genetycznego.

\section{Implementacja}
Do rozwiązania tego zadania posłużyliśmy się językiem Python oraz pakietami \texttt{matplotlib} 
i~\texttt{numpy}. 

\subsection{Reprezentacja}
Jako podstawę rozwiązania utworzyliśmy klasę \texttt{Node} --- jej obiekty reprezentują elementy struktury drzewiastej, która z kolei stanowi reprezentację formuły. \texttt{Node} może być wartością stałą, operatorem jedno lub dwuargumentowym.

Poszukujemy wzoru punkcji pierwotnej dla zadanej funkcji podcałkowej. Oznacza to, że dla każdego ewoluowanego wzoru obliczamy pochodną metodą ilorazu różnicowego. Obliczamy te wartości w 21 punktach kontrolnych, co 5\% wartości przedizału. Następnie porównujemy otrzymane wartości z wektorem wartości funkcji podcałkowej. \textit{Bład} danego osobnika określa się na podstawie sumy wartości absolutnych powstałych różnic.

\subsection{Ewolucja}
Nową populację tworzą pary osobników pochodzące z krzyżowania, które polega na podmianie dwóch, losowo wybranych poddrzew w kopiach rodziców.

W każdej epoce jest pewna szansa, domyślnie $\frac{3}{5}$, że osobnik zostanie wybrany do rozrodu. Największą szansę na przekazanie genów mają osobniki z najmniejszym \textit{błędem}. Jeśli osobnik nie zostanie wybrany do rozrodu, ma pewną szansę, domyślnie $\frac{1}{5}$, na przebycie mutacji, która jednak jest maskowana przed współplemieńcami. Proces wyboru powtarza się, dopóki nowa populacja nie osiągnie rozmiarów starej.

Operator mutacji powoduje losową zmianę w danym osobniku. W $\frac{1}{3}$ przypadków zmianie ulega sama wartość danego węzła, operator lub stała. Następna $\frac{1}{3}$ przypadków to rekurencyjna mutacja jednego z podwęzłów osobnika. Pozostała część to podmiana całej gałęzi na nową, losową. Jeśli dany węzeł nie posiada węzłów potomnych --- jest szansa 50/50 na wybór jednego z pozostałych wariantów.

\section {Wyniki i wnioski}
Końcowym rezultatem działania algorytmu jest wzór funkcji będącej przybliżeniem funkcji pochodnej funkcji wejściowej. Opcjonalnie rysowany jest wykres ukazujący obie funkcje. W przypadku idealnym czerwona krzywa jest wykresem funkcji pochodnej funkcji zaznaczonej kolorem niebieskim.

Wzór najlepszej funkcji można również obejrzeć podczas procesu ewolucji, co 20 iteracji. Jest prezentowana w notacji polskiej, z nadmiarowymi nawiasami, znanymi m. in. z języka LISP.

Dla porównania otrzymanych przez nas wyników ze stanem faktycznym posługiwaliśmy się Wolframem Alpha\footnote{\url{http://http://www.wolframalpha.com/}}.

Za każdym przebiegiem algorytmu zauważamy, że:
\begin{itemize}
  \item 
  \item 
  \item w zadaniu tym zmniejszenie pradopodobieństwa rozrodu pomaga zachować różnorodność genetyczną, która z kolei pozwala procesowi ewolucji wyjść z lokalnych minimów błędów. W praktyce przyjęte przez nas wartości domyślne pomagają uzyskać satysfakcjonujące wyniki.
\end{itemize}

\end{document}
